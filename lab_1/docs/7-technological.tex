\chapter{Технологическая часть}

В данном разделе будут приведены требования к ПО, реализация алгоритмов и средства реализации.

\section{Требования к программному обеспечению}

Входные данные: две строчки из латинских или кириллических символов;

Выходные данные: целое число, которое является искомым расстоянием между двумя строчками.


\section{Средства реализации}

Для реализации был выбран ЯП Python \cite{bib1}.
Выбор обсуловлен наличием функции вычисления процессорного времени в библиотеке
time \cite{bib2}. Время было замерено с помощью функции process\_time().

\section{Реализация  алгоритмов}

В листингах \ref{lst:rec-lev} --- \ref{lst:ddyn-lev} представлены реализации алгоритмов
нахождения расстояний Левенштейна и Дамерау-Левенштейна.

\begin{center}
\captionsetup{justification=raggedright,singlelinecheck=off}
\lstinputlisting[label=lst:rec-lev, firstline=1, lastline=22,
caption=Реализация рекурсивного алгоритма
нахождения расстояния Левенштейна]{../src/lsht.py}
\end{center}


\begin{center}
\captionsetup{justification=raggedright,singlelinecheck=off}
\lstinputlisting[label=lst:dyn-lev, firstline=26, lastline=51,
caption=Реализация динамического алгоритма
нахождения расстояния Левенштейна]{../src/lsht.py}
\end{center}


\begin{center}
\captionsetup{justification=raggedright,singlelinecheck=off}
\lstinputlisting[label=lst:ddyn-lev, firstline=53, lastline=79,
caption=Реализация динамического алгоритма
нахождения расстояния Дамерау-Левенштейна]{../src/lsht.py}
\end{center}


\paragraph*{ВЫВОД} ${}$ \newline
В данном разделе рассмотрены средства реализации, требования к ПО и реализации алгоритмов.


\clearpage
