\begin{center}
    \textbf{ВВЕДЕНИЕ}
\end{center}
\addcontentsline{toc}{chapter}{ВВЕДЕНИЕ}

Конвейерной компьютерной архитектуре уделяется значительное внимание с 1960-х годов.
когда потребность в более быстрых и экономичных системах стала критической.
Достоинство конвейерной обработки состоит в том, что она может помочь согласовать скорости различных подсистем без дублирования стоимости всей задействованной системы.

\textbf{Цель лабораторной работы} --- Получить навык организации параллельных
вычислений на основе нативных потоков.

Для достижения поставленной цели необходимо выполнить следующие задачи:
\begin{itemize}
    \item реализовать алгоритм обработки данных с использованием конвеерной обработки;
    \item создание ПО, реализующего разработанный алгоритм;
    \item провести исследование логов.
\end{itemize}


\section{Входные данные}
Входными данными программы является URL ссылка пагинации.


\section{Выходные данные}
Выходные данные --- база данных $SQLite$, содержащая две таблицы:  $texttable$
(URL, text) и 
$textimage$ (URL, image\_URL).

\clearpage

\section{Тестирование}
В таблице ~\ref{tbl:time_measurements} представлены функциональные тесты для разработанного
программного обеспечения. Все тесты пройдены успешно.

\begin{table}[h]
	\begin{center}
		\begin{threeparttable}
		\captionsetup{justification=raggedright,singlelinecheck=off}
		\caption{Время работы алгоритмов (в секундах)}
		\label{tbl:time_measurements}
                        \begin{tabular}{|p{4cm}|p{4cm}|p{4cm}|p{4cm}|}
                            \hline
                            № Теста & Входные данные & Полученные \ данные & Ожидаемые выходные данные \\
                            \hline
                            1 & https://vkusnye-recepty.ru/page/ & Директория с текстами рецептов &
                            Директория с текстами рецептов \\
                            \hline
                            2 & https://nevkusnye-recepty.ru/page/ --- несуществующая ссылка
                              & Error. Invalid URL: https://nevkusnye-recepty.ru/page/1 &
                            Error. Invalid URL: https://nevkusnye-recepty.ru/page/1 \\
                            \hline
                            3 &  & Error. Invalid URL: 1 &
                            Error. Invalid URL: 1 \\
                            \hline

                        \end{tabular}
		\end{threeparttable}
    \end{center}
\end{table}

\section{Исследование}
В ходе исследования требуется получить характеристики созданного ПО в зависимости
от количества потоков или от количества обрабатываемых страниц сайта.

\subsection{Технические характеристики}
Технические характеристики устройства, на котором выполнялось тестирование:
\begin{itemize}
    \item операционная система: Linux ~\cite{bib1};
    \item память: 16 GB;
    \item процессор: AMD Ryzen 7 5800H ~\cite{bib2, bib3}.
\end{itemize}

\subsection{Исследование логов}
В ходе исследования необходимо сформировать лог обработки задач.
В таблицах \ref{tbl:log1} --- \ref{tbl:log2} приведен лог для 6 задач (до снятия статистики).

\begin{table}[h]
	\begin{center}
		\begin{threeparttable}
		\captionsetup{justification=raggedright,singlelinecheck=off}
		\caption{Лог. Первая часть.}
		\label{tbl:log1}
                    \begin{tabular}{|l|l|}
                        \hline
                        Время & Событие\\
                        \hline
                        2024-11-12 16:07:04.402639978 & Задача 1 вошла в очередь 1\\
                        2024-11-12 16:07:04.402646511 & Задача 130 вошла в очередь 1\\
                        2024-11-12 16:07:04.402650829 & Задача 772 вошла в очередь 1\\
                        2024-11-12 16:07:04.402654716 & Задача 1926 вошла в очередь 1\\
                        2024-11-12 16:07:04.402658654 & Задача 3592 вошла в очередь 1\\
                        2024-11-12 16:07:04.402662731 & Задача 5770 вошла в очередь 1\\
                        2024-11-12 16:07:04.402957190 & Задача 1 вышла из очереди 1\\
                        2024-11-12 16:07:04.402964774 & Задача 1 в 1-м обработчике\\
                        2024-11-12 16:07:04.930509909 & Задача 1 вышла из 1-го обработчика\\
                        2024-11-12 16:07:04.930761837 & Задача 1 вошла в очередь 2\\
                        2024-11-12 16:07:04.930782907 & Задача 130 вышла из очереди 1\\
                        2024-11-12 16:07:04.930784570 & Задача 1 вышла из очереди 2\\
                        2024-11-12 16:07:04.930787545 & Задача 130 в 1-м обработчике\\
                        2024-11-12 16:07:04.930872737 & Задача 1 в 2-м обработчике\\
                        2024-11-12 16:07:05.024848604 & Задача 1 вышла из 2-го обработчика\\
                        2024-11-12 16:07:05.024977738 & Задача 1 вошла в очередь 3\\
                        2024-11-12 16:07:05.025106603 & Задача 1 вышла из очереди 3\\
                        2024-11-12 16:07:05.025228965 & Задача 1 в 3-м обработчике\\
                        2024-11-12 16:07:05.272790526 & Задача 1 вышла из 3-го обработчика\\
                        2024-11-12 16:07:05.536423739 & Задача 130 вышла из 1-го обработчика\\
                        2024-11-12 16:07:05.536633237 & Задача 130 вошла в очередь 2\\
                        2024-11-12 16:07:05.536655589 & Задача 772 вышла из очереди 1\\
                        2024-11-12 16:07:05.536659947 & Задача 772 в 1-м обработчике\\
                        2024-11-12 16:07:05.536723468 & Задача 130 вышла из очереди 2\\
                        2024-11-12 16:07:05.536812246 & Задача 130 в 2-м обработчике\\
                        2024-11-12 16:07:05.631056763 & Задача 130 вышла из 2-го обработчика\\
                        2024-11-12 16:07:05.631196057 & Задача 130 вошла в очередь 3\\
                        2024-11-12 16:07:05.631319110 & Задача 130 вышла из очереди 3\\
                        2024-11-12 16:07:05.631367843 & Задача 130 в 3-м обработчике\\
                        \hline
                    \end{tabular}
		\end{threeparttable}
    \end{center}
\end{table}


\begin{table}[h]
	\begin{center}
		\begin{threeparttable}
		\captionsetup{justification=raggedright,singlelinecheck=off}
		\caption{Лог. Вторая часть.}
		\label{tbl:log2}
                    \begin{tabular}{|l|l|}
                        \hline
                        2024-11-12 16:07:05.885698402 & Задача 130 вышла из 3-го обработчика\\
                        2024-11-12 16:07:06.240815616 & Задача 772 вышла из 1-го обработчика\\
                        2024-11-12 16:07:06.241013511 & Задача 772 вошла в очередь 2\\
                        2024-11-12 16:07:06.241028740 & Задача 772 вышла из очереди 2\\
                        2024-11-12 16:07:06.241029602 & Задача 1926 вышла из очереди 1\\
                        2024-11-12 16:07:06.241033669 & Задача 1926 в 1-м обработчике\\
                        2024-11-12 16:07:06.241049930 & Задача 772 в 2-м обработчике\\
                        2024-11-12 16:07:06.331846487 & Задача 772 вышла из 2-го обработчика\\
                        2024-11-12 16:07:06.331909907 & Задача 772 вошла в очередь 3\\
                        2024-11-12 16:07:06.331983878 & Задача 772 вышла из очереди 3\\
                        2024-11-12 16:07:06.332034934 & Задача 772 в 3-м обработчике\\
                        2024-11-12 16:07:06.687379580 & Задача 772 вышла из 3-го обработчика\\
                        2024-11-12 16:07:07.001586823 & Задача 1926 вышла из 1-го обработчика\\
                        2024-11-12 16:07:07.001653940 & Задача 1926 вошла в очередь 2\\
                        2024-11-12 16:07:07.001682063 & Задача 3592 вышла из очереди 1\\
                        2024-11-12 16:07:07.001682374 & Задача 1926 вышла из очереди 2\\
                        2024-11-12 16:07:07.001687554 & Задача 3592 в 1-м обработчике\\
                        2024-11-12 16:07:07.001704907 & Задача 1926 в 2-м обработчике\\
                        2024-11-12 16:07:07.094612177 & Задача 1926 вышла из 2-го обработчика\\
                        2024-11-12 16:07:07.094820903 & Задача 1926 вошла в очередь 3\\
                        2024-11-12 16:07:07.094959466 & Задача 1926 вышла из очереди 3\\
                        2024-11-12 16:07:07.095097397 & Задача 1926 в 3-м обработчике\\
                        2024-11-12 16:07:07.358947922 & Задача 1926 вышла из 3-го обработчика\\
                        2024-11-12 16:07:07.876209634 & Задача 3592 вышла из 1-го обработчика\\
                        2024-11-12 16:07:07.876422077 & Задача 3592 вошла в очередь 2\\
                        2024-11-12 16:07:07.876440311 & Задача 5770 вышла из очереди 1\\
                        2024-11-12 16:07:07.876443217 & Задача 3592 вышла из очереди 2\\
                        2024-11-12 16:07:07.876444770 & Задача 5770 в 1-м обработчике\\
                        2024-11-12 16:07:07.876560299 & Задача 3592 в 2-м обработчике\\
                        2024-11-12 16:07:07.971533027 & Задача 3592 вышла из 2-го обработчика\\
                        2024-11-12 16:07:07.971676719 & Задача 3592 вошла в очередь 3\\
                        2024-11-12 16:07:07.971808198 & Задача 3592 вышла из очереди 3\\
                        2024-11-12 16:07:07.971857372 & Задача 3592 в 3-м обработчике\\
                        2024-11-12 16:07:08.236796512 & Задача 3592 вышла из 3-го обработчика\\
                        2024-11-12 16:07:08.622002433 & Задача 5770 вышла из 1-го обработчика\\
                        2024-11-12 16:07:08.622235515 & Задача 5770 вошла в очередь 2\\
                        2024-11-12 16:07:08.622254922 & Задача 5770 вышла из очереди 2\\
                        2024-11-12 16:07:08.622278386 & Задача 5770 в 2-м обработчике\\
                        2024-11-12 16:07:08.717779015 & Задача 5770 вышла из 2-го обработчика\\
                        2024-11-12 16:07:08.717833008 & Задача 5770 вошла в очередь 3\\
                        2024-11-12 16:07:08.717958426 & Задача 5770 вышла из очереди 3\\
                        2024-11-12 16:07:08.718004964 & Задача 5770 в 3-м обработчике\\
                        \hline
                    \end{tabular}
		\end{threeparttable}
    \end{center}
\end{table}

\clearpage

В результате проведенного исследования логов было подтверждено, что
конвейерная обработка выполняет различные этапы параллельно, обеспечивая
более высокую эффективность обработки по сравнению с простой последовательной обработкой.


Среднее время жизни  задачи - $5.17$ секунд.
Среднее время ожидания в первой очереди - $1.4$ секунды, когда в остальных очередях нет
простоя. Скорость обработки  задачи - $0.57, 0.11, 0.29$ секунд для
первого, второго и третьего обработчика соответственно.


\newpage

\begin{center}
    \textbf{ЗАКЛЮЧЕНИЕ}
\end{center}
\addcontentsline{toc}{chapter}{ЗАКЛЮЧЕНИЕ}

Цель работы достигнута. Решены все поставленные задачи:
\begin{itemize}
    \item реализован алгоритм обработки данных с использованием конвеерной обработки;
    \item созданано ПО, реализующего разработанный алгоритм;
    \item проведено исследование логов.
\end{itemize}




