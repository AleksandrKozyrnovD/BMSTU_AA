\chapter{Технологическая часть}
В данном разделе будут приведены требования к программному обеспечению,
средства реализации, листинги кода




\section{Требования к программному обеспечению}

Входные данные: матрица $A$ и  $B$, где количество столбцов первой матрицы равно количеству
строк второй матрицы;

Выходные данные: матрица $C$.




\section{Средства реализации}

Для реализации алгоритмов был выбран язык $C$, потому что
можно контролировать количество выделяемой и используемой памяти
в ЭВМ. Это необходимо на ЭВМ с ограниченным количеством оперативной памяти.

\section{Реализация алгоритмов}

В листингах \ref{code:1} --- \ref{code:3} предоставлены листинги реализации
алгоритмов умножения матриц.

\begin{center}
\captionsetup{justification=raggedright,singlelinecheck=off}
\lstinputlisting[label=code:1,firstline=118, lastline=140,
caption=Классический алгоритм умножения матриц]{../src/struct_matrix.c}
\end{center}



\begin{center}
\captionsetup{justification=raggedright,singlelinecheck=off}
\lstinputlisting[label=code:2,firstline=143, lastline=208,
caption=Алгоритм умножения матриц Винограда]{../src/struct_matrix.c}
\end{center}



\begin{center}
\captionsetup{justification=raggedright,singlelinecheck=off}
\lstinputlisting[label=code:3,firstline=210, lastline=273,
caption=Оптимизированный алгоритм умножения матриц Винограда]{../src/struct_matrix.c}
\end{center}

\clearpage

\paragraph*{ВЫВОД} ${}$ \newline

В данном разделе были представлены средства реализации, требования к программному обеспечению,
технические характеристики и реализация алгоритмов.


