\chapter{Аналитическая часть}

В данном разделе будут рассмотрены алгоритмы умножения матриц.

\section{Описание алгоритмов}

Пусть даны матрицы $A$ с размерами $N\times M$ и  $B$ с размерами $M\times K$.
В результате умножения матрицы $A$ на матрицу $B$ получается матрица  $C$ с размером
$N\times K$.


\subsection{Классический алгоритм умножения матриц}


Пусть даны матрицы $A$ размерностью  $n\times m$ и матрица $B$ размерностью  $m\times k$:
\begin{equation}
    A = 
\begin{pmatrix}
    a_{11} & a_{12} & \ldots & a_{1m} \\
    a_{21} & a_{22} & \ldots & a_{2m} \\
    \hdots & \hdots & \ddots & \vdots \\
    a_{n1} & a_{n2} & \ldots & a_{nm}
\end{pmatrix}, \ \
    B = 
\begin{pmatrix}
    b_{11} & b_{12} & \ldots & b_{1k} \\
    b_{21} & b_{22} & \ldots & b_{2k} \\
    \hdots & \hdots & \ddots & \vdots \\
    b_{m1} & b_{m2} & \ldots & b_{mk}
\end{pmatrix}.
\end{equation}

\medskip

Тогда умножением матрицы $A$ на матрицу  $B$ называется, где матрица $C$:
\begin{equation}
    C = A\times B =
\begin{pmatrix}
    c_{11} & c_{12} & \ldots & c_{1k} \\
    c_{21} & c_{22} & \ldots & c_{2k} \\
    \hdots & \hdots & \ddots & \vdots \\
    c_{n1} & c_{n2} & \ldots & c_{nk}
\end{pmatrix},
\end{equation}
где
\begin{equation}
    c_{ij} = \sum\limits_{k=1}^{m} a_{ik} \cdot b_{kj} \ \ (i = \overline{1,n}, \
    j = \overline{1,k}).
\end{equation}


\subsection{Алгоритм Винограда умножения матриц}

Пусть даны матрицы $A$ и  $B$, имеющие размерность  $4\times 4$.
\begin{equation}
    A = 
\begin{pmatrix}
    a_{11} & a_{12} & a_{13} & a_{14} \\
    a_{21} & a_{22} & a_{23} & a_{24} \\
    a_{31} & a_{32} & a_{33} & a_{34} \\
    a_{41} & a_{42} & a_{43} & a_{44} \\
\end{pmatrix}, \ \
    B = 
\begin{pmatrix}
    b_{11} & b_{12} & b_{13} & b_{14} \\
    b_{21} & b_{22} & b_{23} & b_{24} \\
    b_{31} & b_{32} & b_{33} & b_{34} \\
    b_{41} & b_{42} & b_{43} & b_{44} \\
\end{pmatrix}.
\end{equation}

Для получение очередного элемента $c_{ij}$ матрицы $C$ в классическом
алгоритме умножения матрицы выполняется по формуле:
\begin{equation}
    c_{ij} = \begin{pmatrix} a_{n1} & a_{n2} & a_{n3} & a_{n4} \end{pmatrix} 
    \times 
    \begin{pmatrix} b_{j1}\\ b_{j2}\\ b_{j3}\\ b_{j4} \end{pmatrix},
\end{equation}
где $a_{ni}, \ i = \overline{1,4}$ - элементы $n$-ой строки матрицы  $A$;
$b_{jk}, \ k = \overline{1,4}$ - элементы $j$-ого столбца матрицы  $B$.

В алгоритме Винограда для ускорения рассчетов снижается доля дорогих операций (умножения) и
заменой их на сложение. Для достижения этой цели выполняется предварительная обработка.
Запоминаются значения, что позволит заменить некоторые умножения сложением. Таким
образом:
\begin{equation}
    c_{ij} = (a_{n1} + b_{j2}) (a_{n2} + b_{j1})
    + (a_{n3} + b_{j4})(a_{n4} + b_{j 3})
    - a_{n 1}a_{n 2} - a_{n 3}a_{n 4} - b_{j 1}b_{j 2} - b_{j 3}b_{j 4},
\end{equation}
где элементы $a_{n 1}a_{n 2}, \ a_{n 3}a_{n 4}, \ b_{j 1}b_{j 2}, \ b_{j 3}b_{j 4}$ - значения,
которые получаются в предварительной обработке.

\paragraph*{Вывод} ${}$ \newline

В данном разделе были рассмотрены алгоритмы умножения матриц. Основные различия между
алгоритмами - наличие предварительной обработки и количество операций умножения.


