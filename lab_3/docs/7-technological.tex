\chapter{Технологическая часть}

В данном разделе будут приведены требования к программному обеспечению,
реализация алгоритмов и средства реализации.

\section{Требования к программному обеспечению}

Входные данные: Массив и искомое значение;

Выходные данные: Индекс и количество потребованных операций.


\section{Средства реализации}

Для реализации был выбран язык программирования Python \cite{bib2}. Выбор обусловлен наличием библиотеки
$matplotlib$ \cite{bib3}. Для построения графиков была использована функция  $bar$ \cite{bib4}.

\section{Реализация алгоритмов}

В листингах \ref{lst:1} --- \ref{lst:2} представлены реализации алгоритмов.

\begin{center}
\captionsetup{justification=raggedright,singlelinecheck=off}
\lstinputlisting[label=lst:1, firstline=6, lastline=16, caption=Алгоритм нахождения значения
полным перебором]{../src/main.py}
\end{center}


\begin{center}
\captionsetup{justification=raggedright,singlelinecheck=off}
\lstinputlisting[label=lst:2, firstline=18, lastline=35, caption=Алгоритм нахождения значения
с помощью бинарного поиска]{../src/main.py}
\end{center}

\paragraph*{ВЫВОД} ${}$ \\

В данном разделе были реализованы алгоритмы поиска заданного значения в массиве
полным перебором и с помощью двоичного поиска, рассмотрены средства реализации, 
предусмотрены требования к программному обеспечению.

\clearpage
