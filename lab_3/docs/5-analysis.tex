\chapter{Аналитическая часть}

В данном разделе рассмотрены алгоритмы нахождения заданного значения в множестве.


\section{Описание алгоритмов}

Данные алгоритмы используются для поиска заданного значения в множестве.

\subsection{Алгоритм поиска полным перебором}

При использовании алгоритма поиска полным перебором происходит перебор всех
элементов множества \cite{bib0}. Это означает, если искомое значение лежит в начале множества,
то оно будет найдено быстрее, чем если оно лежало в конце множества.


\subsection{Алгоритм нахождения с помощью бинарного поиска}

При использовании алгоритма с бинарным поиском не происходит
перебор всех элементов множества. В этом алгоритме множество должно быть
отсортированным. Вводятся левая и правая граница поиска.
Выбирается центральный элемент в границе поиска и сравнивается с искомым значением.
Если искомое значение меньше центрального элемента, то правая граница множества
передвигается левее центрального элемента. Если
искомое значение больше центрального элемента, то левая граница множества
передвигается правее центрального элемента. Так повторяется, пока искомое
значение не будет равно центральному элементу или область поиска сужается до нуля \cite{bib1}.

\clearpage

\paragraph*{ВЫВОД} ${}$ \\

В данном разделе рассмотрены алгоритмы нахождения заданного значения в множестве.


%\textbf{ВЫВОД}

